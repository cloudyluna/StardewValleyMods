\documentclass[a4paper,12pt]{article}
\usepackage[utf8]{inputenc}
\usepackage{geometry}
\geometry{margin=1in}
\usepackage{hyperref}
\usepackage{graphicx}

\title{SelectiveEating}
\author{Cloudyluna}
\date{\today}

\begin{document}
	
	\section{SelectiveEating by Cloudyluna}
	
	A simple quality of life mod to make eating easier and automatic with configurable settings.
	
	\subsubsection{Features}
	
	\begin{itemize}
		\item Automatically eats food based on health or stamina percentage deprivation.
		\item Forbid certain food from being eaten.
		\item Prioritize and make a list of food as the most important ones to be consumed first.
		\item An option (default) to stay in the same facing direction even after eating.
		\item Increase or decrease eating time interval for roleplaying/challenge or performance reasons.
		\item Options can be configured through \href{https://www.nexusmods.com/stardewvalley/mods/5098}{Generic Config Menu mod} or \texttt{config.json} file within the mod directory.
	\end{itemize}
	
	
	\subsection{Building}
	
	\begin{quote}
		Attention: Make sure you already have set up \$\textit{GAME\_PATH} pointing to your Stardew Valley game folder first.
	\end{quote}

	If you cloned this project from the Github and have \texttt{nix} installed, run \texttt{nix develop} to load the development environment. Then run \texttt{make} or \texttt{dotnet build} to build the project.
	
	\begin{verbatim}
		# Example to build SelectiveEating for release
		git clone https://github.com/cloudyluna/StardewValleyMods
		cd SelectiveEating
		make release
	\end{verbatim}
	
	\subsection{URL link to the repository}
	\href{https://github.com/cloudyluna/StardewValleyMods/tree/main/SelectiveEating}{Github repository}
	
	\subsection{Thanks to}
	\begin{itemize}
		\item \href{https://github.com/Pathoschild/SMAPI}{SMAPI dev and contributors} for making Stardew Valley modding accessible.
		
		\item \href{https://www.nexusmods.com/stardewvalley/mods/5098}{Generic Config Menu dev and contributors} for making mod configuration through GUI, simple and easy.
	\end{itemize}
	
	\subsection{License}
	
	\begin{itemize}
		\item This project is licensed under the AGPL-3.0-or-later. See the \texttt{LICENSE} file for details.
		
		\item  The bundled FSharp.Core is licensed under the MIT license. See \texttt{sublicenses/FSharp.MIT} file for details.
	\end{itemize}
	
	
\end{document}